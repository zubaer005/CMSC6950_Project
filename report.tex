\documentclass{article}
\usepackage{graphicx}
\usepackage{float}
\bibliographystyle{unsrt}

\title{CMSC 6950 Final Project - argopy}
\author{Kabir Zubaer}

\begin{document}
\maketitle

\section{Introduction}

The objective of this project was to select an open source software package and perform computational tasks using data obtained from a package of choice. The scientific package chosen for my project is called argopy \cite{maze2020argopy}. argopy is a Python library that can be used to download, analyze and interpret ocean data collected by Argo floats. In particular, the argopy Python library permits users to obtain Argo float measurements from Argo floats worldwide that measure pressure, temperature and salinity of the worlds oceans from the surface to 2000m depth every 10 days. Traditionally, the large number of files, data variables and use of jargon associated with the Argo data often accompanied a challenging workflow, especially for new users. The motivation of argopy was to provide a Python friendly library where Argo float data can be easily accessible and readable for users who are new to and/or experts with Argo float data. For this project, two computational tasks have been carried out using data extracted from the argopy Python library and other Python modules such as numpy, pandas, geopandas and matplotlib. 

    
 

\section{Conclusions}

In conclusion, two computational tasks were performed using data extracted from the argopy Python library. The first computational task provided a visualization of total Argo float numbers through time using the pandas, matplotlib and geopandas Python libraries. The second computational task consisted of extracting data from two Argo floats using argopy. The trajectory and temperature variation with time was then visualized by also using the pandas, matplotlib and geopandas Python libraries.

\bibliography{refs}

\end{document}